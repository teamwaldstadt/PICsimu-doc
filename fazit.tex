\chapter{Fazit}

Das Projekt der Entwicklung eines PIC-Simulator hat sehr zum Verständnis von den Komponenten eines Mikrocontrollers beigetragen. Aufgrund von bisherigen Erfahrungen und Kenntnissen entschieden wir uns für die objektorientierte Programmiersprache Java.

Die Umsetzung in verschiedenen Klassen war sehr übersichtlich und strukturiert, sodass es kaum Probleme mit der Programmarchitektur gab. Die Darstellung jedes Kommandos in einer eigenen Klasse war zwar aufwendig, trug aber dazu bei, dass die Befehle übersichtlich und klar organisiert waren. Zudem wurde dadurch das objektorientierte Design konsequent umgesetzt.

Die Oberfläche mittels der Java Swing-Bibliothek ohne ein grafisches Fenstererstellungs-Tool ist ebenfalls gelungen, sodass am Ende sogar ein responsives Design mit allen nötigen Funktionen implementiert werden konnte. Alles in allem kann gesagt werden, dass sich die Sprache Java für die Entwicklung dieses Projektes gut eignet.

Der Quellcode dieses Projekts kann unter \href{https://github.com/teamwaldstadt/PICsimu/}{https://github.com/teamwaldstadt/PICsimu} eingesehen werden.
